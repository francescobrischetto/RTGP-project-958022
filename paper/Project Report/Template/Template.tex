\documentclass[11pt,a4paper]{article}
%%%%%%%%%%%%%%%%%%%%%%%%% PACKAGE starts HERE %%%%%%%%%%%%%%%%%%%%%%%%
\usepackage{graphicx}
\usepackage{caption}
\usepackage{microtype}
\captionsetup[table]{name=Tabel}
\captionsetup[figure]{name=Gambar}
\usepackage{tabulary}
% \usepackage{minted}
% \usepackage{amsmath}
\usepackage{fancyhdr}
% \usepackage{amssymb}
% \usepackage{amsthm}
\usepackage{placeins}
% \usepackage{amsfonts}
\usepackage{graphicx}
\usepackage[all]{xy}
\usepackage{tikz}
\usepackage{verbatim}
\usepackage[left=2cm,right=2cm,top=3cm,bottom=2.5cm]{geometry}
\usepackage{hyperref}
\hypersetup{
    colorlinks,
    linkcolor={red!50!black},
    citecolor={blue!50!black},
    urlcolor={blue!80!black}
}
\usepackage{caption}
\usepackage{subcaption}
\usepackage{multirow}
\usepackage{psfrag}
\usepackage[T1]{fontenc}
\usepackage{lmodern}
\usepackage[scaled]{beramono}
% Enable inserting code into the document
\usepackage{listings}
\usepackage{xcolor} 
% custom color & style for listing
\definecolor{codegreen}{rgb}{0,0.6,0}
\definecolor{codegray}{rgb}{0.5,0.5,0.5}
\definecolor{codepurple}{rgb}{0.58,0,0.82}
\definecolor{backcolour}{rgb}{0.95,0.95,0.92}
\definecolor{LightGray}{gray}{0.9}
\lstdefinestyle{mystyle}{
	backgroundcolor=\color{backcolour},   
	commentstyle=\color{green},
	keywordstyle=\color{codegreen},
	numberstyle=\tiny\color{codegray},
	stringstyle=\color{codepurple},
	basicstyle=\ttfamily\footnotesize,
	breakatwhitespace=false,         
	breaklines=true,                 
	captionpos=b,                    
	keepspaces=true,                 
	numbers=left,                    
	numbersep=5pt,                  
	showspaces=false,                
	showstringspaces=false,
	showtabs=false,                  
	tabsize=2
}
\lstset{style=mystyle}
\renewcommand{\lstlistingname}{Kode}
%%%%%%%%%%%%%%%%%%%%%%%%% PACKAGE ends HERE %%%%%%%%%%%%%%%%%%%%%%%%


%%%%%%%%%%%%%%%%%%%%%%%%% Data Diri %%%%%%%%%%%%%%%%%%%%%%%%
\newcommand{\student}{\textbf{Fill Name Here (And Nim Here)}}
\newcommand{\course}{\textbf{Course Name (Course Code)}}
\newcommand{\assignment}{\textbf{xxx}}



%%%%%%%%%%%%%%%%%%% using theorem style %%%%%%%%%%%%%%%%%%%%
\newtheorem{thm}{Theorem}
\newtheorem{lem}[thm]{Lemma}
\newtheorem{defn}[thm]{Definition}
\newtheorem{exa}[thm]{Example}
\newtheorem{rem}[thm]{Remark}
\newtheorem{coro}[thm]{Corollary}
\newtheorem{quest}{Question}[section]
%%%%%%%%%%%%%%%%%%%%%%%%%%%%%%%%%%%%%%%%
\usepackage{lipsum}%% a garbage package you don't need except to create examples.
\usepackage{fancyhdr}
\pagestyle{fancy}
\lhead{Student Name in Header (Student Number in Header)}
\rhead{ \thepage}
\cfoot{\textbf{Assignment title is typed here}}
\renewcommand{\headrulewidth}{0.4pt}
\renewcommand{\footrulewidth}{0.4pt}

%%%%%%%%%%%%%%  Shortcut for usual set of numbers  %%%%%%%%%%%

\newcommand{\N}{\mathbb{N}}
\newcommand{\Z}{\mathbb{Z}}
\newcommand{\Q}{\mathbb{Q}}
\newcommand{\R}{\mathbb{R}}
\newcommand{\C}{\mathbb{C}}
\setlength\headheight{14pt}
%\renewcommand{\contentsname}{NewContent}  If you want to change name of table of content
%%%%%%%%%%%%%%%%%%%%%%%%%%%%%%%%%%%%%%%%%%%%%%%%%%%%%%%%%%
\begin{document}
\thispagestyle{empty}
\begin{center}
	\includegraphics[scale = 0.15]{Figure/ifitera-header.png}
	\vspace{0.1cm}
\end{center}
\noindent
\rule{17cm}{0.2cm}\\[0.3cm]
Name: \student \hfill Assignment To: \assignment\\[0.1cm]
Course: \course \hfill Date: Date\\
\rule{17cm}{0.05cm}
\vspace{0.1cm}

\pagebreak

\tableofcontents
\pagebreak

%%%%%%%%%%%%%%%%%%%%%%%%%%%%%%%%%%%%%%%%%%%%% BODY DOCUMENT %%%%%%%%%%%%%%%%%%%%%%%%%%%%%%%%%%%%%%%%%%%%%
\section{First Header}
    To use headers, you simply create $\backslash${\tt{section}} in your script section. The numbering in the header will be automatically generated. If you need a line break, you can add two slashes with a backslash like this $\backslash\backslash$

\section{How to Include a Link}
    You can start learning to write in \LaTeX by writing actual assignments or reports. At first it may seem difficult. But if you practice diligently, writing in LaTeX will get you used to it and even more fun than writing in Microsoft Word.\\
    If you want to include a link in \LaTeX you can do so with the $\backslash${\tt{href}} command, for example in the following link \href{http://aldi_tob_2000.staff.gunadarma.ac.id/Downloads/files/17359/Membuat+dokumen+dengan+latex.pdf}{(Sumber: PDF Tutorial Belajar Latex)}.
    
\subsection{Sub Header or Sub Chapter}
     UTo use subheaders, you can simply create $\backslash${\tt{subsection}} or even $\backslash${\tt{subsubsection}} for sub-sections. Then how to create subheaders or even headers but without numbering? Check the information below.

\subsection*{Header Without Numbering}
     Untuk menggunakan subheader tanpa penomoran, anda cukup membuat $\backslash${\tt{subsection*}} atau bahkan  $\backslash${\tt{subsubsection*}} maupun $\backslash${\tt{section*}}.
    
\subsection{Bold, Italic, Plaintext}
\begin{itemize}
    \item You can make bold with $\backslash${\tt{textbf}} commands like \textbf{the following}.
    \item You can make italics with $\backslash${\tt{textit}} commands like \textit{the following}. \item To underline, just type the command $\backslash${\tt{underline}} like \underline{the following}.
    \item To create a list like this one, use the $\backslash${\tt{item}} command between
\end{itemize}

\subsection{Code Snippets}
    The following is an example of using $\backslash${\tt{begin{lstlisting}}} to write a code snippet. In this case I use Python language. If you are using C or something else, just adjust the parameters in the $\backslash${\tt{begin{lstlisting}}} section. You can see it in the code snipptes \ref{labelkode}
    
    \begin{lstlisting}[language=Python, caption=Write the caption here
    class,label={labelkode}]
    
    class SynthiaDataset(Dataset):

    CLASSES = [
        "void", "road", "sidewalk", "building", "wall", "fence", "pole", "traffic light", 
        "traffic sign", "vegetation", "terrain", "sky", "person", "rider", "car", "truck", 
        "bus", "train", "motorcycle", "bicycle", "road lines", "other", "road works"
    ]
    
    def __init__(self, path="../SYNTHIA-SF", classes=None, augmentation=None, preprocessing=None, valid=False):
        self.rootdir = Path(path)
        self.data_imgs, self.data_gts = self.prepare_data(valid,path)
        self.valid = valid

    
        if classes == None:
            classes = self.CLASSES 
        self.class_values = [self.CLASSES.index(cls.lower()) for cls in classes]
        
        self.augmentation = augmentation
        self.preprocessing = preprocessing
    \end{lstlisting}

\section{Loading Multi-Image}
The following is an example of how to load multiple images.

\begin{figure}[h]
	\centering
	\begin{subfigure}[b]{0.4\textwidth}
		\centering
		\def\svgwidth{\columnwidth}
		\includegraphics[width=1\textwidth]{Figure/ifitera-header.png}
		\caption{Augment Result 1}
		\label{fig:aug-1}
	\end{subfigure}
	\qquad %add desired spacing between images, e. g. ~, \quad, \qquad, \hfill etc. 
	%(or a blank line to force the subfigure onto a new line)
	\begin{subfigure}[b]{0.4\textwidth}
		\centering
		\def\svgwidth{\columnwidth}
		\includegraphics[width=1\textwidth]{Figure/ifitera-header.png}
		\caption{Augment Result 2}
		\label{fig:aug-2}
	\end{subfigure}
	\caption{Augmentation Samples}\label{fig:aug}
\end{figure}


\newpage
\section{Loading Image}
If you don't want to load multiple images, aka one caption contains only one image. Please imitate the following method. If you feel that your image is out of place, learn how to place an image on the \href{https://www.overleaf.com/learn/latex/Positioning_of_Figures}{following link}.
\begin{figure}[h]
    \centering
    \includegraphics[width=0.8\textwidth]{Figure/ifitera-header.png}
    \caption{Ini Captionnya}
    \label{fig:my_label}
\end{figure}

\section{Loading Table}
This part is the part that I think is quite difficult. But you can use the help of table designers on the internet, for example \href{https://www.tablesgenerator.com}{TablesGenerator} or \href{https://www.latex-tables.com}{Latex-Tables }. There is also a plugin for microsoft excel named \href{https://ctan.org/tex-
	archive/support/excel2latex?lang=en}{CTAN}, but I rarely use it. An example of a table can be seen in the following \ref{tab-sample} table.

\begin{table}[h]
\caption{Contoh Tabel}
\label{tab-contoh}
\centering
\resizebox{12cm}{!}{%
\begin{tabular}{ccccccccc}
\hline
\multirow{2}{*}{\textbf{Exp}} & \multirow{2}{*}{\textbf{Mask}} & \multicolumn{3}{c}{\textbf{GT}}            & \multicolumn{3}{c}{\textbf{Proposed}}      & \multirow{2}{*}{\textbf{RMSE}} \\ \cline{3-8}
                              &                                & \textbf{Avg} & \textbf{Max} & \textbf{Min} & \textbf{Avg} & \textbf{Max} & \textbf{Min} &                                \\ \hline
\multirow{2}{*}{1}            & Yes                            & 89.60        & 114.84        & 70.31        & 89.14        & 118.45        & 68.24        & 3.66                           \\
                              & No                             & 90.55        & 112.50        & 75.00        & 89.03        & 109.75        & 71.35        & 3.60                            \\ \hline
\multirow{2}{*}{2}            & Yes                            & 109.84        & 125.98       & 98.44       & 108.62       & 121.30       & 98.26      & 4.04                           \\ 
                              & No                             & 106.62       & 123.44        & 96.09       & 106.48       & 122.19       & 93.37        & 3.95                           \\ \hline
3                             & Yes                            & 74.42        & 94.92       & 62.99        & 73.49       & 102.32        & 60.43       & 3.27                         \\ \hline
Mean                          &                                & 90.61         & 114.34        & 80.57        & 89.70       & 114.80       & 78.33        & 3.63                           \\ \hline
\end{tabular}
}
\end{table}

\section{References and Bibliography}
Here's the slightly \textit{tricky} part. You must include your bibliography in a file with a .bib extension on the left side of this overleaf. You can easily export this dot bib content, whether it's from Google Scholar, Mendeley, or other citation management.
How to use it is quite easy.
For example, now I want to cite one of the existing documents, for example wikipedia, I just write $\backslash${\tt{cite}} which contains the cite-key of the entry in file.bib \cite{Wikipedia_contributors2021-bb}. Another example of writing citations is as follows \cite{Name2018-hd}
\newpage
\bibliographystyle{IEEEtran}
\bibliography{Referensi}
\end{document}